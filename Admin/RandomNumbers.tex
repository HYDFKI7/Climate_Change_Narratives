\documentclass{article}\usepackage[]{graphicx}\usepackage[]{color}
%% maxwidth is the original width if it is less than linewidth
%% otherwise use linewidth (to make sure the graphics do not exceed the margin)
\makeatletter
\def\maxwidth{ %
  \ifdim\Gin@nat@width>\linewidth
    \linewidth
  \else
    \Gin@nat@width
  \fi
}
\makeatother

\definecolor{fgcolor}{rgb}{0.345, 0.345, 0.345}
\newcommand{\hlnum}[1]{\textcolor[rgb]{0.686,0.059,0.569}{#1}}%
\newcommand{\hlstr}[1]{\textcolor[rgb]{0.192,0.494,0.8}{#1}}%
\newcommand{\hlcom}[1]{\textcolor[rgb]{0.678,0.584,0.686}{\textit{#1}}}%
\newcommand{\hlopt}[1]{\textcolor[rgb]{0,0,0}{#1}}%
\newcommand{\hlstd}[1]{\textcolor[rgb]{0.345,0.345,0.345}{#1}}%
\newcommand{\hlkwa}[1]{\textcolor[rgb]{0.161,0.373,0.58}{\textbf{#1}}}%
\newcommand{\hlkwb}[1]{\textcolor[rgb]{0.69,0.353,0.396}{#1}}%
\newcommand{\hlkwc}[1]{\textcolor[rgb]{0.333,0.667,0.333}{#1}}%
\newcommand{\hlkwd}[1]{\textcolor[rgb]{0.737,0.353,0.396}{\textbf{#1}}}%
\let\hlipl\hlkwb

\usepackage{framed}
\makeatletter
\newenvironment{kframe}{%
 \def\at@end@of@kframe{}%
 \ifinner\ifhmode%
  \def\at@end@of@kframe{\end{minipage}}%
  \begin{minipage}{\columnwidth}%
 \fi\fi%
 \def\FrameCommand##1{\hskip\@totalleftmargin \hskip-\fboxsep
 \colorbox{shadecolor}{##1}\hskip-\fboxsep
     % There is no \\@totalrightmargin, so:
     \hskip-\linewidth \hskip-\@totalleftmargin \hskip\columnwidth}%
 \MakeFramed {\advance\hsize-\width
   \@totalleftmargin\z@ \linewidth\hsize
   \@setminipage}}%
 {\par\unskip\endMakeFramed%
 \at@end@of@kframe}
\makeatother

\definecolor{shadecolor}{rgb}{.97, .97, .97}
\definecolor{messagecolor}{rgb}{0, 0, 0}
\definecolor{warningcolor}{rgb}{1, 0, 1}
\definecolor{errorcolor}{rgb}{1, 0, 0}
\newenvironment{knitrout}{}{} % an empty environment to be redefined in TeX

\usepackage{alltt}


\title{Random Numbers for Student Submissions}
\author{Marc Los Huertos}
\IfFileExists{upquote.sty}{\usepackage{upquote}}{}
\begin{document}

\maketitle


\section*{Random Numbers}

Please use one of the random numbers assigned to you for each assignment. You can use the assign numbers in any order and for any assignment you like. 

% latex table generated in R 3.5.2 by xtable 1.8-4 package
% Wed Sep 11 14:05:09 2019
\begin{table}[ht]
\centering
\begin{tabular}{rlrrrrr}
  \hline
 & First & 1 & 2 & 3 & 4 & 5 \\ 
  \hline
1 & Pauline & 25124 & 26659 & 45070 & 42893 & 14398 \\ 
  2 & Miranda & 61909 & 60346 & 15041 & 86408 & 35477 \\ 
  3 & Aminah & 37384 & 38979 & 26090 & 30518 & 25175 \\ 
  4 & Kate & 90359 & 91789 & 90905 & 57948 & 34070 \\ 
  5 & Santi & 22644 & 64226 & 42554 & 35846 & 41438 \\ 
  6 & Emily T & 28636 & 83339 & 22991 & 76771 & 82901 \\ 
  7 & Lucy & 57975 & 66354 & 36257 & 57808 & 28379 \\ 
  8 & Jonathan & 52294 & 11790 & 48726 & 93267 & 44160 \\ 
  9 & Kai & 26869 & 16772 & 40252 & 46156 & 61296 \\ 
  10 & Alexandra & 98159 & 87098 & 20158 & 24972 & 39605 \\ 
  11 & Aurora & 17276 & 77793 & 88994 & 40169 & 86021 \\ 
  12 & Alice & 69317 & 58319 & 22397 & 57409 & 50000 \\ 
  13 & Emma & 20231 & 24631 & 53296 & 74131 & 39550 \\ 
  14 & Dawn & 43761 & 26783 & 73329 & 25799 & 74279 \\ 
  15 & Mike & 83631 & 26791 & 21218 & 21465 & 34088 \\ 
  16 & Howard & 12601 & 95523 & 64499 & 29156 & 96887 \\ 
  17 & Emily P & 53437 & 25191 & 13265 & 94338 & 31353 \\ 
   \hline
\end{tabular}
\end{table}


Question: What are the chances that two or more people might have the same random number?  Answering this is something taught in a probability course, which provides a foundation for all statistics.
\end{document}
