\documentclass{article}\usepackage[]{graphicx}\usepackage[]{color}
% maxwidth is the original width if it is less than linewidth
% otherwise use linewidth (to make sure the graphics do not exceed the margin)
\makeatletter
\def\maxwidth{ %
  \ifdim\Gin@nat@width>\linewidth
    \linewidth
  \else
    \Gin@nat@width
  \fi
}
\makeatother

\definecolor{fgcolor}{rgb}{0.345, 0.345, 0.345}
\newcommand{\hlnum}[1]{\textcolor[rgb]{0.686,0.059,0.569}{#1}}%
\newcommand{\hlstr}[1]{\textcolor[rgb]{0.192,0.494,0.8}{#1}}%
\newcommand{\hlcom}[1]{\textcolor[rgb]{0.678,0.584,0.686}{\textit{#1}}}%
\newcommand{\hlopt}[1]{\textcolor[rgb]{0,0,0}{#1}}%
\newcommand{\hlstd}[1]{\textcolor[rgb]{0.345,0.345,0.345}{#1}}%
\newcommand{\hlkwa}[1]{\textcolor[rgb]{0.161,0.373,0.58}{\textbf{#1}}}%
\newcommand{\hlkwb}[1]{\textcolor[rgb]{0.69,0.353,0.396}{#1}}%
\newcommand{\hlkwc}[1]{\textcolor[rgb]{0.333,0.667,0.333}{#1}}%
\newcommand{\hlkwd}[1]{\textcolor[rgb]{0.737,0.353,0.396}{\textbf{#1}}}%
\let\hlipl\hlkwb

\usepackage{framed}
\makeatletter
\newenvironment{kframe}{%
 \def\at@end@of@kframe{}%
 \ifinner\ifhmode%
  \def\at@end@of@kframe{\end{minipage}}%
  \begin{minipage}{\columnwidth}%
 \fi\fi%
 \def\FrameCommand##1{\hskip\@totalleftmargin \hskip-\fboxsep
 \colorbox{shadecolor}{##1}\hskip-\fboxsep
     % There is no \\@totalrightmargin, so:
     \hskip-\linewidth \hskip-\@totalleftmargin \hskip\columnwidth}%
 \MakeFramed {\advance\hsize-\width
   \@totalleftmargin\z@ \linewidth\hsize
   \@setminipage}}%
 {\par\unskip\endMakeFramed%
 \at@end@of@kframe}
\makeatother

\definecolor{shadecolor}{rgb}{.97, .97, .97}
\definecolor{messagecolor}{rgb}{0, 0, 0}
\definecolor{warningcolor}{rgb}{1, 0, 1}
\definecolor{errorcolor}{rgb}{1, 0, 0}
\newenvironment{knitrout}{}{} % an empty environment to be redefined in TeX

\usepackage{alltt}
%\documentclass{tufte-handout}
\usepackage{hyperref}
\usepackage{wasysym}
<<<<<<< HEAD
\usepackage{soul}

=======
>>>>>>> ed07ab52056d861d1292c2734178cf1058cbeb8f
\newenvironment{itemize*}%
  {\begin{itemize}%
    \setlength{\itemsep}{0pt}%
    \setlength{\parskip}{0pt}}%
  {\end{itemize}}
	
\newenvironment{enumerate*}%
  {\begin{enumerate}%
    \setlength{\itemsep}{0pt}%
    \setlength{\parskip}{0pt}}%
  {\end{enumerate}}

\title{How are climate data used by activists?}
\author{Marc Los Huertos}
%\date{}
\IfFileExists{upquote.sty}{\usepackage{upquote}}{}
\begin{document}

\maketitle
\tableofcontents

\newpage
\section{Introduction}

\subsection{Climate and the IPCC}

According the the Inter-Governmental Panel on Climate Change or IPCC, the last three decades at the Earth's surface have seen the most amount of successive warming than any decades since 1850. All in all, the averaged data for ocean surface and land temperatures combined points to a rise of 0.85 [0.65 to 1.06] degrees Celsisus  from 1880 to 2012 \footnote{IPCC,  2014: Climate  Change  2014:  Synthesis  Report
. Contribution  of  Working  Groups  I,  II  and  III  to  the  Fifth  Assessment  Report  of  the Intergovernmental Panel on Climate Change [Core Writing Team, R.K. Pachauri and L.A. Meyer (eds.)]. IPCC, Geneva, Switzerland, 151 pp.} -- but this global average is not evenly distributed accross the globe. 
%find rate of change and insert citation

This change and causes of this change are perhaps some of the most contested environmental issues in the 50 year history of the environmental movement. So much so, that as EA students, we need to understand what the scientific conclusions are and how these conclusions were made, while understanding the potential implications.

\subsubsection{What is the IPCC?}

The Intergovernmental Panel on Climate Change (IPCC) is a scientific and intergovernmental body under the auspices of the United Nations, set up at the request of member governments and dedicated to the task of providing the world with an objective, scientific view of climate change and its political and economic impacts.

The IPCC was created in 1988. Initially it was set up by the World Meteorological Organization (WMO) and the United Nations Environment Program (UNEP) to prepare assessments on all aspects of climate change and its impacts, based on available scientific information. The goals of the IPCC is to formulate realistic response strategies. 

\subsubsection{IPCC's Role}

The role of the IPCC is to assess on a comprehensive, objective, open and transparent basis the scientific, technical and socio-economic information relevant to understanding the scientific basis and risk of human-induced climate change, its potential impacts and options for adaptation and mitigation.

As an intergovernmental body, membership of the IPCC is open to all member countries of the United Nations (UN) and WMO. Currently 195 countries are Members of the IPCC.

The IPCC has published five comprehensive assessment reports reviewing the latest climate science (Table \ref{tab:IPCC}), as well as a number of special reports on particular topics. These reports are prepared by teams of relevant researchers selected by the Bureau\footnote{I am note sure what this means, but haven't had the time to sort it out!  Suggestions?} from government nominations.\footnote{I'd be interested to see how this process is done in the USA.} Drafts of these reports are made available for comment in open review processes to which anyone may contribute.

\begin{table}
\caption{Major IPCC Reports}\label{tab:IPCC}
\centering
\begin{tabular}{lc}\hline
Assessment Report             & Published \\ \hline\hline
First Assessment Report (FAR) & 1990    \\
Supplementary Report          & 1992   \\
Second Asessment Report (SAR) & 1995    \\
Third Assessment Report (TAR) & 2001    \\
Fourth Assessment Report (AR4)& 2007   \\
Fifth Assessment Report (AR5) & 2014  \\
Sixth Assessment Report (AR6) & 2022*  \\ \hline
\end{tabular}
\end{table}

Each assessment report is in three volumes, corresponding to Working Groups I, II, and III. Unqualified, ``the IPCC report'' is often used to mean the Working Group I report, which covers the basic science of climate change.

\subsection{Global and Regional Average Temperature Changes}

In speaking about the topic of climate change it is easy to cite a global temperature average. However, this is part of what makes climate change such a contentious issue. An average temperture increase for the globe is actually somewhat abstract and, perhaps, beyond what humans can reliably perceive. In this sense, perhaps we should evaluate how temperature (and/or rainfall) might be changing at regional scales. 

Are there strategies to help us appreciate the impact of climate change on weather patterns at the regional level? Can regional level impacts help develop politically viable strategies to address the global problem? How do local activists consider regional data and do their interpretations align with currrent scientific thinking?

%Thus, for this project, we'll try to understand \textbf{how temperature changes ``map'' onto a community that we care about}. To do this we will obtain and analyze tempterature data to determine if weather changes have compelling impacts on local communities.

In other words, we are seeking to answer the question: how do activitists use regional climate change patterns?

\subsection{Goals of this Document}

This document is meant to be a resource and guide to you as you undertake the task of answering the question: how do activitists use regional climate data? This document contains:

\begin{enumerate*}
  \item Descriptions of the overarching goals and approaches for each assignement in the project;
  \item Guidelines and resources for completing each assignment; and
  \item Grading rubrics and descriptions of how we will evaluate the project process and products.
\end{enumerate*}

\section{Project Description}

\subsection{Driving Question(s)}

The driving questions for this project can be stated as follows: 

\begin{itemize*}
%  \item How do activitists use regional climate change patterns?
  \item How is regional climate data being used by cliamte activists?
\end{itemize*}

However, as we move along in the course and in this project you may find it worthwhile to try phrasing the questions in a number of ways -- this might help you find ways to make the question more provactive and interesting, For example, instead of asking ``how is the climate in my region changing?" you could ask ``how does the changing climate in my region affect cloud coverage in my area and what ecological impact does that have on nearby forests/wildlife?"

You can modify these questions to develop the project that you might find compelling.

%In addition, we may develop ``sub-questions'' whose answers might inform the main question or questions. For example, 

%\begin{itemize*}
%  \item Are there biases in weather data? Can these biases be corrected? If so, how?
%  \item How can we evaluate trends? What are the most appropriate statistical tools to test for trends?
%  \item What is the best way to display visual data?  Are there best practices to guide a public product to make it more compelling or interactive?
%\end{itemize*}

\subsection{Public Products}

Science is a social project. From the questions we ask, to the results and their presentation, science is usually embedded in a culture of norms (such as research journals, reports, documentaries, etc.). To frame our science within these norms of communication, each of us will publish a series of blogs utilizing our findings to answer our driving question.

In addition, each student will write and submit an Op-Ed piece to a regional newspaper that frames regional climate issues into a newsworthy item.

%Finally, we will hold a Q \&A session with public school teachers to help them implement NGSS standards on weather and climate.

\section{Directed Practice}

\subsection{Learning Goals}

For this project, you will obtain weather records to answer the driving questions. However, the exact way you decide to answer the question is largely up to you. Nevertheless, specific skills and knowledges will be required to successfully address the question:  

\paragraph{Skills}

\begin{itemize*}
  \item Ability to obtain and process weather long-term weather records;\footnote{I advise students to find stations with atleast 50 years of data.}
  \item evaluate temporal trends in weather data;
  \item research the environmental impacts on human or non-human communities; and
  \item communicate conclusions to the public with special attention to guide how data misinterpretations should be considered.
  \item determine how data are used by activists.
\end{itemize*}

\paragraph{Knowledge}
\begin{itemize*}
  \item Understand how data climate data is currated;
  \item Analyze climate impacts from around the world.
\end{itemize*}

Throughout this project, we will co-develop the strategies and skills to address this question and help you make some conclusions and present the results to the public.

\subsection{Resources}

Students will have the following tools available:

\begin{itemize*}
  \item Servers where stored weather data can be downloaded;
  \item R Studio Server with some scripts \& libraries to help develop analyses;
  \item Github to store project codes and as a platform to make the product public;
  \item Lectures, reports, and presentations on climate change science, the social and ecological implications of climate change, and public policy and politics of climate change;
  \item \href{https://github.com/marclos/Climate_Change_Narratives/raw/master/Admin/RandomNumbers.pdf}{Random numbers for student submissions}; and
  \item Shiny app templates that might be used as a container for interactive content.\footnote{Currently under-development -- We will likely skip this application since I not confident in using this particular tool.}
\end{itemize*}

\subsubsection{Software Guides}

Much of the environmental data collected has become electronic. Using software requires certain skills, which requires students to appreciate that different types of software exist. 

In particular, I am constantly thinking critically about what software I advise students to use and learn. For my ``developmental thinking'' on this issue, I suggest you read the following draft white paper: \href{https://github.com/marclos/Climate_Change_Narratives/raw/master/Admin/Liberation_via_Open_Source_Software.pdf}{Open Source and Liberation}. 

Since climate data rely on large time series datasets, we need to rely on software to access to and process these data, we need use tools to access, pre-process, and analyze these data. Below are resources that we have developed to assist you in this class (Table \ref{tab:softwareguides}).

\begin{table}[h]
\caption{Software guides developed for EA30. These SOPs have been developed by students and faculty over the years and are loaded on the github.com/SOPs repository.}\label{tab:softwareguides}
\centering
\begin{tabular}{ll}\hline
SOP \#    & Description                                 \\\hline\hline
06        & \href{https://github.com/marclos/Climate_Change_Narratives/blob/master/Analysis_SOPs/SOP06_Rstudio_Server_Github.pdf}{An Introduction to Rstudio and Github} \\
06b       & Introduction to Markdown--Html  \\
06c       & Introduction to Markdown--Word  \\
06d       & Visual Display of Data using R  \\
\hline
\end{tabular}
\end{table}

\subsubsection{Data Processing and Analysis Tools}

Much of the environmental data collected has become electronic. Thus, to access to and process these data, we need use tools to access, pre-process, and analyze these data using computer software. 

Below are resources that we have developed to assist you in this class (Table \ref{tab:tools}).

\begin{table}[h]

\caption{Resources to obtain, pre-process, and analyze NOAA climate data.}\label{tab:tools}
\centering
\begin{tabular}{ll}\hline
Step \#    & Title                                 \\\hline\hline
1        & \href{https://github.com/marclos/Climate_Change_Narratives/raw/master/Analysis_Resources/1_Obtaining_Climate_Records.pdf}{Obtaining Climate Records}\\
2        & \href{https://github.com/marclos/Climate_Change_Narratives/raw/master/Analysis_Resources/2_Using_NOAA_Climate_Records.pdf}{Using NOAA climate Records}\\
3        & \href{https://github.com/marclos/Climate_Change_Narratives/raw/master/Analysis_Resources/3_Evaluating_Monthly_Trends_CHCNDaily.pdf}{Evaluating Monthly Trends using CHCN-Daily}\\
%4        & \href{https://github.com/marclos/Climate_Change_Narratives/raw/master/Analysis_SOPs/SOP90_Analyzing_Trends.pdf}{Analyzing Climate Trends} \\ 
\hline
\end{tabular}
\end{table}

These SOPs can be found in the Rproject/Github Respistory --Climate Change Narratives and in the 'Analysis\_SOPs' directory.

The analysis of trend data can range from simple to complex. For a brief introduction, read an introduction on the \href{https://climatedataguide.ucar.edu/climate-data-tools-and-analysis/trend-analysis}{Trend Analysis} on the Climate Data Guide website.

\subsubsection{Readings and Other Climate Change Resources}

I have put these readings in the syllabus schedule, since these readings are more background material. 

\subsubsection{Contested Science and Critical Thinking}


\begin{itemize*}
  \item \href{https://github.com/marclos/Climate_Change_Narratives/raw/master/Communication_Resources/Logical_Fallacies.pdf}{``The Rhetorical Tools of Logical Fallacies''}
  
  \item \href{https://github.com/marclos/Climate_Change_Narratives/raw/master/Communication_Resources/Critical_Thinking.pdf}{``Critical Thinking in EA''}
\end{itemize*}

\subsubsection{Communication Resources}

We will learn and practice our skills to communicate using written and oral media. 

Scientific writing is a skill that takes years to develop. Although there are many types of readings, scientific writing does have some unique characteristics that will seem a bit awkward. However, you might be surprised about how much you already know about technical writing. We have selected key resources that we think will help you further develop and improve your writing skills.

However, specific genres require specific adjustments in our writing style. Please use the following to help in your writing process:\footnote{I have used various emails and conversations to produce the resources below, but they are still rough around the edges. I usually hire students to improve these resources after I get them started -- let me know if this is something you might be interested in doing after the semester ends.}

\begin{itemize*}
  \item \href{https://github.com/marclos/Climate_Change_Narratives/raw/master/Communication_Resources/Writing_About_Climate.pdf}{``Scientific Writing and Climate Narratives''}

  \item \href{https://github.com/marclos/Climate_Change_Narratives/raw/master/Communication_Resources/Op-Ed_Guidelines.pdf}{``Op-Ed Guidelines''}
  
  \item \href{https://github.com/marclos/Climate_Change_Narratives/raw/master/Communication_Resources/Scientific_Blog_Guidelines.pdf}{``Scientific Blog Guidelines''}
  
  \item \href{https://github.com/marclos/Climate_Change_Narratives/raw/master/Communication_Resources/Visualing_Data.pdf}{``Visual Presentation of Data using R''}
  
  \item \href{https://github.com/marclos/Climate_Change_Narratives/raw/master/Communication_Resources/Citing_Sources.pdf}{``Citing References in EA30''}
  
  \item \href{https://github.com/marclos/Climate_Change_Narratives/raw/master/Communication_Resources/Peer_Review-Dos_and_Donts.pdf}{``Peer review writing -- Dos and Don'ts''}
\end{itemize*}

Oral presentations will also be part of this project and course. Students will use Rpres for their presentations and here is a short tutorial for this tool:

\begin{itemize*}
  \item \href{https://github.com/marclos/Climate_Change_Narratives/raw/master/Communication_Resources/TBD.pdf}{``Using Rpres to Develop Oral Presentations''}
  
  \item \href{https://github.com/marclos/Climate_Change_Narratives/raw/master/Communication_Resources/TBD.pdf}{``Guide for Oral Presentations''}
  
  \item \href{https://github.com/marclos/Climate_Change_Narratives/raw/master/Communication_Resources/TBD.pdf}{``Guide to Make Effective Video Presentations for Covid-19''} Coming soon! \smiley{}
\end{itemize*}


Below is my list of key areas to be cognizant to improve our capacity to communicate science:

\begin{description}
  \item[Clarity, Forthrightness, and Economical]
  \item[Accuracy and Precision] Accuracy and precision occurs at several scales in writing, word choice, sentence level, paragraph, and essay level. 
  \item[Critical Thinking]
  \item[Cited Evidence]
\end{description}



\section{Project Milestones}

To complete the project in a timely fashion, we will be adhering to a rather strick schedule (Table \ref{tab:milestones}).

\begin{table}[h]

\caption{Project Deliverables, milestones and point distribution. *I encourage students to continue to improve their blogs (and their grades) even after they are published.}
\label{tab:milestones}
\begin{tabular}{lcll}\hline
Deliverable                 & Launch    & Due Date  & Points \\\hline\hline
Draft Blog Template         & Sept 4    & {\color{red}Sept 12}   & 5 \\
Climate Science Expert Team Video & Sep 1 & Sep 12   & 20 \\
Op-Ed \#1                   & Aug 28    & {\color{red}Sep 19}    & 10 \\
Draft Regional Analysis     & Sep 1     & {\color{red}Sep 19}   & 20 \\
Regional Climate Literature Review  & Sep 1   & {\color{red}Sep 19}   & 20\\
Climate Science Expert Team Video -- Forum    & Sep 1  & Sep 19     & 15 \\
%Regional Climate Literature Review -- Peer Review & Oct 1   & Oct 3 & 10 \\
Blog Draft {\color{red}Text and Figures} & Sep 1   & Sep 26    & 50 \\
Blog -- Peer Review         & Sep 1    & Oct 3    & 10\\
Published Blog*              & Sep 1    & Oct 10   & 50 \\
%Op Ed \#2 Draft            & Oct 9     & Oct 13    & 10 \\
Op Ed \#2 Submission        & Sep 1    & Oct 10    & 25 \\ \hline
\end{tabular}
\end{table}

\section{Op Ed \#1: Scientific Values and Climate Activists}

%\subsection{Op-Ed 1}

\subsection{Rationale}

Climate change may be the most controversial environmental issue in history. However, compared to other issues, this history is relatively short. Fueled by opposing political parties and industry goals, the conclusions of scientists is a fundamental source of conflict -- thus, science itself has become extremely politicized. 

Nevertheless, how and where science and scientists became embroiled became a battle ground negotiating the appropriate level of regulation (regulatory reach), economic and industrial \textit{Laissez-faire}, and environmental risks. Environmental issues are almost always controversial and in the case of climate change, few dominate the political agenda like climate change. 

Nevertheless, in a pandemic and election year, the political states may be higher than normal. The role of activism in the US (and world) has changed in recent years with a higher reliance on social media. Of course, with the current requirements for Covid-19 social distance, social media might be the primary source of information for many. 

In this context, we need to determine the role of and changes in activism and it's role in climate science. Moreover, we need to evaluate the how activist are using scientific information, especially climate records as they promote their agendas.

\subsection{Assignment}

Write an Op-Ed piece that describe the role of activism in climate change, their use of climate data during a pandemic and election year. %outlines why residents in a specific US region should care about temperature changes. 
Spend sometime deciding what is currently in the news that you consider a compelling issue to your audience.

\subsection{Submission Format and Naming Convention}

Submit your Op-Ed as a pdf via \texttt{Sakai}, using the following naming convention:

\begin{center}
\textbf{Op-Ed\_1\_XXXXX.pdf},
\end{center}

\noindent using one of your 5 digit random numbers for the Xs. See \url{https://github.com/marclos/Climate_Change_Narratives/raw/master/Admin/RandomNumbers.pdf} to get the list of assigned random numbers. 

\subsection{Grading}

The Op-Ed will be graded using several criteria. First, the topic must be compelling -- connecting current affairs to the historical issues of climate. Second, the Op-Ed should rely on several sources of evidence and citations, while creating fluid prose that compel the reader to continue reading. If the reader gets stuck in statistics or technical jargon, it can be like wading in mud -- but without some ``numbers'' the argument may become glittering generalities without a sense of a gritty reality. Again, your job is to find a compelling balance. Finally, you want the read to jump out of their seat and ``do something''. Thus, the Op-Ed should compel the reader into action, see assignment handout for more information.

%\subsection{Readings}






\section{Draft Blog Template}

The assignment is pretty simple. Can you create an Rmd file, with your title and name as the author, some text of what you are going to be writing about, one image, and a simple plot. {\color{red}Note: You may have already done this already, you don't have to do this again. Howver, we suggest you make make some changes to keep the blog moving forward -- for example, if you had daily data, create monthly means. If you had TMAX, check out TMIN. If you have TMIN, you might consider snow pack or precipitation.  

Push the Rmd and html file into github and name the file with your surname. It would be super helpful to get the naming conventions sorted out.}

\begin{center}
X.Rmd and X.hmtl
\end{center}

\noindent where X is your surname. {\color{red} Once you push the template into R, you are done, no Sakai submission necessary.}

\section{Developing Specialized Knowledge}

To develop expertise, we will rely on teams of students to develop and evaluate various aspect of climate data. Each of us form an essential component for the effort. Organized as teams and expert groups, we will disassemble the project into chunks that each of us will contribute in specific and effective ways. This expertise will be used to develop our Q \& A sessions, as well as, to help us develop and write our op-ed and blogs. The experts should include areas of contravery and how scientists and non-scientists wressle over the data.

\subsection{Topics of Expertise}

We will will create expert groups on to present the following topics:

\begin{enumerate*}
  \item Radiative Gases -- What are they and what do they do?
  
List the major compounds categorized as radiative gases and describe how various processes determine their role as GHGs. Provide detail on how different wavelengths of light interact with the gases. Finally, a discussion of water is key, since it is one of the main sources of controversy. 
  
  \item GHG Emission Trends and Sources -- Carbon Dioxide (CO$_2$), Nitrous Oxide (N$_2$O), and Methane (CH$_4$).

Describe how carbon dioxide and other GHGs are emitted and remain in the atmosphere. Distinguish between natural and anthropogenic sources and why that distinction might be important. Desribe various type of sources and how these might be linked to certain types of economic development and activities. In addition, describe the role of vegetation and other forms of carbon sequestration. Describe the sources of uncertainty and the common arguments that are used to discount the role of greenhouse gases (e.g. carbon dioxide is natural and can't be a pollutant, humans exhale carbon dioxide, carbon has been higher in the past, etc). 

  \item Role of Water and Other Feedbacks
  
Climate change feedback is important in the understanding of global warming because feedback processes may amplify or diminish the effect of each climate forcing, and so play an important part in determining the climate sensitivity and future climate state. Feedback in general is the process in which changing one quantity changes a second quantity, and the change in the second quantity in turn changes the first. Positive feedback amplifies the change in the first quantity while negative feedback reduces it. Be sure to include the following feedbacks: Clouds, gas release (Methane is a big one), ice-albedo, carbon, and water vapor. Describe the uncertainties and how some of these have become politicized.

  \item Terrestrial Surface Temperature Records
  
The instrumental temperature record provides the temperature of Earth's climate system from the historical network of in situ measurements of surface air temperatures and ocean surface temperatures. Data are collected at thousands of meteorological stations, buoys and ships around the globe. The longest-running temperature record is the Central England temperature data series, that starts in 1659. The longest-running quasi-global record starts in 1850. 
  
  \item Ocean Temperatures and Sea Level

In recent decades more extensive sampling of ocean temperatures at various depths have begun allowing estimates of ocean heat content but these do not form part of the global surface temperature datasets. Describe how ocean temperatures have been measured over time and how these have lead to a range of interpretations of the results. Discuss how the thermal expansion of water may influence sea leval rise. Discuss how sea temperature change may affect different parts of the world differently. Describe the methods to distinguish sea level rise and coastal elevation changes, including how satellites work to collect these data. Describe the areas of uncertainty and how various groups frame these uncertainties.

  \item Satellite-based Temperature Measures
  
Satellites can be used to measure outgoing radiation. However, each atmospheric layer has different properties and is impacted by GHGs in differing ways. Describe how the satelite data has been used, how these instruments have changed and why there are several different methods to evalaute satellite data. Because satellite data has result results, describe how these methods have been used to support or limit our confidence in climate change. Describe sources of uncertainty and how various groups have used the uncertainties to make arguments for and against anthropogenic climate change.
  
  \item Weather Extremes Trends Explained
  
Weather and climate extremes such as hurricanes, tornadoes, heavy downpours, heat waves, and droughts affect all sectors of the economy and the environment, impacting people where they live and work. As usual some claim that more extreme weather has been caused climate change, while others claim that there has been a reduction in extreme events. Please describe why the analyses have not developed into a clear conclusion. 
  
\end{enumerate*}

\subsection{Expert Teams}

Although most of the work will be individual, we will also work in pair for the presentation. Using \href{https://docs.google.com/spreadsheets/d/1VkY8-js4QNeYz6gHXNzKOwVvB2EUkp7Y2BwjRrsfxOk/edit?usp=sharing}{this Google Sheet}, sign up for a topic and as the slots are filled, I will update this document. 

{\color{red}The following students have been assigned to the teams below:

% latex table generated in R 3.6.0 by xtable 1.8-4 package
<<<<<<< HEAD
% Sun Sep 27 09:17:13 2020
=======
% Fri Sep 18 10:36:26 2020
>>>>>>> ed07ab52056d861d1292c2734178cf1058cbeb8f
\begin{table}[ht]
\centering
\begin{tabular}{lll}
  \hline
Topic & Team\_Members & Presentation\_Date \\ 
  \hline
1 & Marc ,  -- ,  -- & 09/12/20 \\ 
  2 & Christina ,  Lila ,  Sarah & 09/12/20 \\ 
  3 & Viviana ,  Bryan ,  Nora & 09/12/20 \\ 
  4 & Jacob ,  Claire L ,  Tramy & 09/12/20 \\ 
  5 & Olivia ,  Owen ,  Claire M & 09/12/20 \\ 
  6 & Anna ,  Melia ,  Katy & 09/12/20 \\ 
  7 & Nikodem ,  Emma ,  Isabel & 09/12/20 \\ 
   \hline
\end{tabular}
\end{table}

}
\subsection{Climate Science Review {\color{red}Videos}}

% subsection{Climate Science Presentation}

\subsubsection{Rational}

Climate change science is complex and requires a tacit understanding of a range of scientific disciplines. Instead of trying to learn all about them, we will hear presentations from our peers on various topics based on their own research. 

Following the adage, 'the best way to learn is to teach', is an appropriate way to think about this assignment. 

\subsubsection{Assignment}

Create a 10-12 minute presentation where each team member should limit their presentation to 3-4 minutes each. Tem minutes goes quickly, so I suggest you practice a few times to ensure that you don't lose unnecessary points. Longer presentations will be penalized.  

\noindent Assignment: 
\begin{itemize*}
  \item Describe the historical development of the scinece/topic.
  \item Describe how data are collected and used to develop conclusions.
  \item Describe areas of uncertainty.
  \item Make an organized presentation that effectively communicates how various scientific arguments have been distorted and politicized;
  \item Identify how conventional scientific standards have been comprimised; and
  \item Use the allotted time (10-12 min) effectively. I suggest you practice, 10 mintues can go very quickly when presenting complext scientific data.\footnote{If your group needs extra time, please send a note on the Slack Channel and we'll decided how to proceed.}
\end{itemize*}

\subsubsection{Submission Format and Naming Convention}

%In addition, each team will present (via open-source software, i.e. rPres) their results to the class. Learning to use Rpres is pretty easy, however, making it do things that we want is another story. 

I have created a link on vidgrid for you to create the video.\footnote{I need to figure out if you can do a group video -- so I'll be checking on that soon!} Please submit and name the vido using the following naming convention: 

\noindent Y\_Topic\_Title,

where "Y" is the topic number enumerated in the previous section. 

%Communications\_Resources in a file called: Introducing\_rPres.md. Open this file and you will find a preview button where the knit button has been in other files. 

%I suggest you open this file and then save a version in your own directory (under Student\_Folders) with the following name -- ``Expert\_Topic.Md, '' where topic is some one word description of the content.

\subsubsection{Presentation Grading Criteria}

The Climate Science Presentation will be grading using the criteria in Table \ref{tab:presentationgrading}.

\begin{table}[h]
\centering
\caption{Presentation Grading Criteria}
\label{tab:presentationgrading}
\begin{tabular}{llp{2.5in}}\hline
Standard            & Percent & Criteria \\ \hline\hline    
Accuracy            & 20\%    & Was the information accurate?\\
Completeness        & 20\%    & Were important issues not addressed? Or important aspects left out?\\
Clarity             & 20\% & Was the presentation clear and logically constructed?\\
Timeliness          & 20\% & Was the presentation completed within the alotted time? \\
Use of Technology   & 20\% & Was technology used effectively?\\ \hline
\end{tabular}
\end{table}

%\subsection{Climate Science Review}

%% subsection{Climate Science Review}

\subsubsection{Rational}

Learning about climate science requires students to dig deep into meteorology, atmospheric chemistry, physics, biogeochemistry, etc. Thus, learning about the science requires a bit of patience and lots of hard work. 

Using a literature genre, this asssignment has been designed to give you the capacity to be familiar and develop some expertise in one topic of climate change and to become a resource to others in the class.

\subsubsection{Assignment}

Investigate the assigned topic so you can write a thoughtful and critical review of the topics. Be sure to include how data might be used to counter common arguments that critique climate change science. Submit a written summary of your research findings and their references. The paper should be less than 4-5 pages (double spaced). 

Please include the following sections:

\begin{itemize*}
  \item Historical Development.
  \item Data Sources and Methods of Analysis.
  \item Areas of Uncertainty.
  \item Politicization of Evidence and Conclusions;
\end{itemize*}
 
\subsubsection{Submission Format and Naming Convention}

Submit via \texttt{Sakai} using the following naming convention: Climate\_Science\_Review\_F17\_XXXXX.pdf, where the XXXXX refer to the five digit \href{https://github.com/marclos/Climate_Change_Narratives/raw/master/Admin/RandomNumbers.pdf}{assigned random numbers}.

\subsubsection{Climate Science Review Grading}

The Climate Science Review will be grading using Table~\ref{tab:climatesciencegrading}. 

\begin{table}[h]
\caption{Climate Science Grading Standards.}
\label{tab:climatesciencegrading}
\begin{tabular}{lll}\hline
Criteria                & Standard      & Percent \\
\hline\hline
Historical Development  & Completness   & 20\% \\
Data Sources and Methods  & Completeness  & 20\% \\
Logical Fallacies & Identified and Confronted   & 20\% \\
Critical Thinking & Advanced                    & 20\%\\
% Graphics          && 20\%\\
Writing           && 20\% \\
Submission        && 20\% \\
\hline
\end{tabular}
\end{table}



\subsection{Climate Science Presentation Forums}

%\subsection{Peer Review of Climate Science Report}

\subsubsection{Rational}

Communicating about climate sciene is fraught with potential stumbling blocks. First, it's hard to hit the audience knowledge level correctly. Second, many readers have biases, which means that readers have filters that we might not be able to appreciate. Finally, since we are not climate scientists, we are working to translate the science into a languages that others can understand -- back to first point!  

By using peer reveiw, we can develop methods that might reduce this stumbling blocks, where your peers will be able to read and evaluate if the text is clear, accurate, and comprehendable. 

\subsubsection{Assignment}

Read two of your peers' Climate Report (see Table~\ref{tab:peerreviewassignments}) and evaluate the report using the following questions as a guide:

\begin{enumerate}
  \item What is the argument of the report?
  \item Describe three sources of evidence for this argument. What is compelling about these sources? What might be criticisms about these sources? If evidence is missing please explain.
  \item What counter arguments are used in the report? Are these accurately characterized? Is the evidence for the counter arguments documented and evaluated to your satisfaction?  If not what's missing?
  \item List terms that may not be defined well enough for you and the general public. 
  \item Re-write two sentences that could be improved for clarity.
  \item Re-write two sentences that could be improved for precision.
  \item Re-write two sentences that could be improved for accuracy.
\end{enumerate}

\begin{table}[h]
\caption{Peer Review Assignments -- Based on Random Numbers.}
\label{tab:peerreviewassignments}
\begin{tabular}{lcc}\hline
Reviewer    &   To be Reveiwed   & To be Reviewed\\ 
\hline\hline
Kirara      & 11790   & 66354\\
Ally        & 98159   & 11790 \\
Caroline    & 61296   & 98159 \\
Meily       & 91789   & 22644 \\
Valentina   & 66354   & 61296 \\
Bebe        & 22644   & 91789 \\
\hline
\end{tabular}
\end{table}

\subsubsection{Submission Format and Naming Convention}

Climate\_Science\_PeerReview\_XXXXX.pdf

\subsubsection{Peer Review Grading}

Table~\ref{tab:peerreviewgrading} will be used to evaluate the peer review exercise. 

\begin{table}[h]
\caption{Peer Review Grading Standards.}
\label{tab:peerreviewgrading}
\begin{tabular}{lll}\hline
Standard                      &   Percent   & \\ 
\hline\hline
Identifying Argument          &   5\%      & \\
Evidence Evaluation           &   10\%      & \\
Counter Argument Evaluation & 5\%    & \\
List of Terms               & 20\%    & \\
Re-write -- Clarity         & 20\%    & \\
Re-write -- Precision       & 20\%    & \\
Re-write -- Accuracy        & 20\%    & \\

\hline
\end{tabular}
\end{table}



\section{Regional Climate Analysis}

Each of us will select a region of interest. Perhaps, somewhere that you have spent a compelling time in or that you wish to know more about. It is ideal if you select something that hasn't been done in previous years, but this isn't crtical. I prefer someplace that you have a connection if that fits your interests better. %Please select a region that has not been done by previous classes. 

\subsection{Analysis of Regional Data}

%\subsection{Analysis of Regional Data}

\subsubsection{Rationale}

Learning to analyze data requires a range of skills that include collecting, analyzing, and interpreting data. For our purposes, this portion of the class is what might traditionally understood as ``doing science.'' We will learn how to test a hypothesis and what it means if we reject the null hypothesis. We will create figures that can be used to communicate our results and finally, we interpret the results using the station data we already collected.

Ultimately, this analysis will be used a template for our blogs and inform our second Opinion Editorials. 

\subsubsection{Assignment}

This assignment provides an avenue to delve into the weather records that you have already obtained. Using the resources supplied, it will be up to you to download, pre-process, and analyze a trend analysis using R -- where the slope, r$^2$, and probability are calculated\footnote{We will have to learn what these are to be able to explain our results! Be sure to ask lots of questions about the statistics so you appreciate this important topic that nearly every scientific field relies!} and explained. 

{\color{red}To support the development of our blogs, this assignment will help us learn as much as we can about our data and better appreciate the long-term trends might be. We suggest you analyze your data by looking at the monthly means and evaluating if the trends from one or two months are particularly strong. In addition, you might consider looking at TMIN, PRCP, or SNW as response variables.

We suggest you save your blog as described below, so you don't have to start over, but can start removing commands that you don't need and explore the data without cluttering up your blog.} 

Using R studio, analyze a long-term climate record, create 3-4 figures that will be used to communicate these climate records, e.g. 100-year temperature \textbf{and} precipitation record for a specific region. Be sure to include language about the ``null'' hypothesis for your trend analysis. 

%\begin{enumerate*}
%  \item Download and analyze data (i.e. make inferences) to create an public product; %I have uploaded all the climate data on a network drive, \url{//fargo/classes/EA30-LosHuertos}{//fargo/classes/EA30-LosHuertos}.\footnote{I haven't been able to get the directory working consistently, so stay tuned on this.}
  
%\end{enumerate*}
% that describes the methods (data sources), data quality, and trends. 


\subsubsection{Submission Format and Naming Convention}

As specified by the milestones (Table \ref{tab:milestones}), submit the draft analysis and results using Rstudio. 

The Rmd file (and the compiled html) should be saved the the in your own directory using the following naming convention:

\begin{center}
\textbf{Region\_XXXXX.Rmd} and \textbf{Region\_XXXXX.html}
\end{center}

\medskip \noindent where XXXXX refer to one of your random numbers. NOTE: Be sure the file still compiles.

Since the regional analysis has been down within Rstudio, you will use the version control procedures to commit and push your analysis onto the Github repository. Thus, be sure to commit and push your files so we have access to the files. {\color{red} Having access to your files gives us a sense of the code you used and if there are easy things that might be used to solve particular issues. Besides pushing the data, export the html file and submit to \texttt{Sakai} for grading.} 

\subsubsection{Data Analysis Grading}

The Data Analysis html files will be grading using the criteria in Table \ref{tab:datagrading}. {\color{red}NOTE: I removed the ``validated model'' criteria because I didn't introduce you to the topic until the week of the 20th. Nevertheless, it will be critical that you do this for your blog, but you might need help from us to do it.}

\begin{table}[h]
\caption{Summary of Data Analysis grading standards.}
\label{tab:datagrading}
\begin{tabular}{llc}\hline
Criteria            &   Standard    & Percent \\ \hline\hline
Records  & Compelling, e.g. Over 60 years & 10\% \\
Knowledge of Data & Limitations and Methods of Collection & 10\% \\
Analysis & p-values and $R^2$ reported  & 20\% \\
{\color{red}\st{Analysis}}          & {\color{red}Validated Model}     & {\color{red}20\%} \\
Interpretation    & Accurate, e.g. rejected null   & 10\% \\
Graphics          & Publishable Quality & 20\% \\
Accessible        & Pushed and named correctly & 10\% \\
\hline
\end{tabular}
\end{table}

\subsection{Regional Climate Literature Review}

%\subsection{Regional Climate Literature Review}

\subsubsection{Rationale}

By using peer reviewed literature, we can assess our regional analysis to determine if the are trends that have been predicted that align or possibly contradict our analysis. 

This assignment is designed to help you put create a blog and have compelling {\color{red}and scientifically rooted information}. 

\subsubsection{Assignment}

Review regionally relative results and conclusions from peer reviewed climate science. %See this document as a resource.


Evaluate peer-reviewed articles to determine potential ecological, economic, and sociological implications of climate patterns. We suggest that you find 3-5 peer review literature articles, but you might fin{\color{red}d} useful government or NGO reports can also be added. {\color{red}In general, peer reviewed literature is considered the best source of scientific information.} 

Summarize these papers into a stand-alone paper. 

\subsubsection{Submission Format and Naming Convention}

The paper should be double-space, 12 point font, and less than 5 pages (excluding citations). As a pdf, the paper should be submitted via Sakai with the following naming convention:

\begin{center}
RegionalImpacts\_F20\_XXXXX.pdf
\end{center}

\medskip \noindent where the XXXXX refer to one set of the assigned random numbers. 

\subsubsection{Grading of the Regional Impacts Summary}

The regional impacts review will be grading using the criteria in Table \ref{tab:regionalimpactsgrading}.

\begin{table}[h]
\caption{Summary of Data Analysis grading standards.}
\label{tab:regionalimpactsgrading}
\begin{tabular}{llc}\hline
Criteria            &   Standard    & Percent \\ \hline\hline
Sources     & Compelling, e.g. cite {\color{red}3-}5 properly cited peer reviewed papers & 40\% \\
Ecological  & Cite knowns and/or unknown impacts            & 20\% \\
Economic    & Describe costs and benefits              & 20\% \\
Social      & Summarize social concerns, e.g. Social Justice             & 20\% \\
Communication    & Accurate, e.g. rejected null   & {\color{red}0\%} \\

\hline
\end{tabular}
\end{table}


\section{Communicating Science}

%\subsection{Analyzing Prior Communication Edeavors}

%\subsubsection{Climate Change Blogs and Websites}

%As we get ready to produce our own blogs, it can be useful to identify models that we can use to emulate or avoid!  

%We have already reviewed previous \href{https://marclos.github.io/Climate_Change_Narratives/}{EA 30 Blogs}, but we might also look to other examples as well. Below is a list of some additional examples of climate blogs:

%\begin{itemize*}
%  \item \href{http://www.accuweather.com/en/weather-blogs/climatechange}{Accuweather}
%  \item \href{http://blogs.nature.com/climatefeedback/}{Nature Magazine}
%  \item \href{https://thinkprogress.org/tagged/climate}{Think Progress}
%  \item \href{http://climateofourfuture.org/}{Climate Four Future}
%\end{itemize*}

%Useful sites: 

%\begin{itemize*}
%  \item \href{http://www.climatecentral.org/news/the-heat-is-on}{Climate Central}
%  \item 
%\end{itemize*}

%\subsubsection{Climate Blogs/Websites Evaluation}

%%\subsubsection{Evaluating Climate Blogs/Websites}

\subsubsection{Rational}

\subsubsection{Assignment}

\subsubsection{Submission Format and Naming Convention}

Review previously written \href{https://marclos.github.io/Climate_Change_Narratives/}{EA 30 Blogs} to evaluate which ones are effective and what you like about each one. 

Select 4-5 blogs and write a summary for each one, describe three things that you like about each one and describe one thing you might improve. Finally, look up one topic for each one that you are more interested in learning and summarize what you find.

\subsection{Writing a Scientific Blog}

%Coming soon! \smiley{}

%\subsection{Writing a Scientific Blog}

\subsubsection{Rational}

{\color{red}Writing our blogs give us the opportunity to evaluate the climate trends in a specific region -- importantly, we are writing the blogs to explain historic and predicted climate change impacts to a specific audience (of your choosing). Thus, we will have to 1) understand climate science enough to explain it, and 2) translate the impacts into something that your readers will appreciate. Thus, as in many environmental issues, learning the scientific issues and figuring out how to communicate these requires patience and practice.


\subsubsection{Developing a Narrative}

Linking your ``Regional Analysis'' and ``Regional Climate Science Literature Review'', compose a blog with the following characteristics: 

\begin{itemize}
  \item focused narrative the engages an audience that is invested in the region analyzed; 
  \item figures that demonstrate climate trends using R and report (and explain) key statistics; and  
  \item cite peer review literature as additional sources of evidence.
\end{itemize}

For the 2020 blog, we want to link these blogs to ``climate activist''. We are using this term as losely as we can. Activitist might include politicians, business leaders, NGO spokepersons, and, of course, activitists at-large. I wonder if we might scour social media as a possible source as well, but as the project develops, we will see what avenues are fruitful. These are your projects, so I suspect I will be surprised by directions you decide to take. 

\subsubsection{Additional Suggestions}

Write blog to effectively and clearly describe results. The blog shall be publish-ready and include the following: 

\begin{itemize*}
  %\item Appropriate and thoughtful statistical analysis;
  \item Describe the economic, cultural, and physical geography of the region (2-3 sentences);
  \item Describe typical climate patterns (1-2 sentances);
  \item Describe where the data were obtained and summarize how the data were processed and analzyed;
  \item Time series plots of temperture data using R (1-3 graphs, with several setences describing the results) as needed by the narrative;
  \item Evaluation of data to determine if trends exists;
  \item Compare results to model predictions/forecasts and possible ecological and economic implications to the region; 
    \item description of what the data tells about about the region, 
  \item a few sentances describing how data can be interpretted; pitfalls of unintentional and intentional misinterpretations; and 
  \item narrative that describes the climate and climate change implications for a community that you care about.
  %\item Describe how the data should be presented, e.g. how the data should be interpreted, and how to avoid misinterpretations that are present in the popular culture.
\end{itemize*}

\subsubsection{Draft and Revision Process}

Science writing (all writing) is a social process. In the case of writing about climate change, we rely on a pretty technical language, which means translating this language for others is not trivial. 

To help in the process, we will turn in a draft blog that will be evaluated by your peers and faculty with the aim to help you focus your thesis, ensure you have described the methods and results effectively, include the evidenced need, and finally in a manner that a broad audience can appreciate. 

\subsubsection{Submission Format and Naming Convention}

To facilite the publishing the blogs, please save the Rmd and html using the following conventions: 

\begin{center}
Surname.Rmd
\end{center}

\noindent Knit the file using the option to create a word document. Upload this document into \texttt{Sakai} on the due date. 

\subsubsection{Scientific Blog Grading}

The Blogs will be grading using Table \ref{tab:bloggrading}. NOTE: We will focus on the mechanics of the blog for the first draft to ensure we have a good foundation on the data and some leaway as we refine our narratives. 



\subsection{Blog Peer Review Process}

%Coming soon! \smiley{}

%\subsection{Peer Review Blogs}

\subsubsection{Rational}

Although writing seen as an individual process -- in reality it's a very social activity, at minumum we don't just write for ourselves. Moreover, as a public product, we want our blogs to be understood and appreciated by a wide audience. 

Reviewing a public product is a priviledge. And for the `reviewed' it's a gift. Thus, for each, the reviewer and reviewed, the value for the greater good is indisputable. 

As you review your collegeues work, try to keep in mind that you are promoting a better outcome and better science. In addition, pay attention to thinks that might have escaped your own process and that you find yourself saying, ``wow, that's a cool approach!''  Perhaps, you might adapt some of the things you read into your own writing!

\subsubsection{Assignment}

To assess the Blogs, each student will review three blogs and submit a evaluation form for each one.

As you review the blog, determine the target audience and evaluat the blog based on your perception of this target. The author may not have the same target in mind, but identifying the divergence might be very helpful. 

Evaluate what if the questions was clearly identified, determine if the data were analyzed clearly (hypothesis, statistical tests), and decide if the conclusions were clear. In addition, make suggestions about how each blog could be improved.

Finally, as we can all appreciate, we are often our own worst readers when it comes to finding issues with grammar and style -- typos often hard to detect and grammar issues may be invisible because we might ``correct'' the text in our brain. Thus, as you read your colleagues blogs and try to identify these issues in the text. Usually, we can do this using Microsoft Word using ``Track Changes''. However, with differential access to Microsoft products, we will have to be creative. If you do not have access to Microsoft, we will work with you to come up with ways to give detailed comments without using this software.

The form will be available on Sakai

\subsubsection{Submission Format and Naming Convention}

Fill out the forms and save them using the following convention:

\begin{center}
BlogAuthorSurname\_XXXXX.pdf
\end{center}

\noindent after submitting the review to \texttt{Sakai}, I will make each available to the Blog authors to help each us revise and improve our blogs. 

\subsubsection{Blog Peer Review Grading}

The peer review process will be graded using Table \ref{tab:blogpeerreviewgrading}. 

\begin{table}[h]
\caption{Blog Peer Review Grading Standards.}
\label{tab:blogpeerreviewgrading}
\begin{tabular}{llr}\hline
Standard          &   Percent   & \\ \hline\hline
Acknowledge specific blog  successes & 25\% \\
Make concrete and detailed suggestions & 75\% \\
\hline
\end{tabular}
\end{table}



\subsection{Publishing Revised Blog}

%Coming soon! \smiley{}

%\subsection{Publishing Revised Blog}

\subsubsection{Rational}

Our capacity to publish our blogs demonstrates that our projects have value beyond our classroom. In addition, these provide a litmus test for our work -- how will the public or specific stakeholders respond to our efforts. Will they see this a valueable, value-added, or problematic?  Although we might not get immediate feedback, the process to publish our blogs gives an opportunity that would be missing if we only wrote papers for the instructor!


\subsubsection{Assignment}

Capitalizing on the regional data analysis and impact summary, create a blog that describes the patterns of clilmate change and their implications. Your final products should include:

\begin{itemize*}
  \item Effectively display climate patterns from NOAA repositories, with at least 6 decades of data. Be sure all graphics are appropriate labeled and have captions that the reader can use to intrepret the data;
  \item Analyze the data using a linear model using R (i.e. lm);
  \item Describe the methods used to obtain and analyze the data; and
  \item Evaluate peer review literature to determine potential regional impacts from climate change -- be sure to include ecological and economic impacts; 
  \item Cite instances of how various scientific arguments have been distorted and politicized;
  \item Identify how conventional scientific standards have been compromised and how arguments that might be based on distortions can be countered.
\end{itemize*}

If it helps, read the Project\_Report.pdf on the Project Site for some helpful hints.

\subsubsection{Submission Format and Naming Convention}

The Blog will be published online (via \url{Github.com}), using the following naming convention: Lastname.Rmd and Lastname.html. 

\subsubsection{Published Blog Grading}

The Blogs will be grading using Table \ref{tab:bloggrading}. 

\begin{table}[h]
\caption{Climate Science Blog Grading.}
\label{tab:bloggrading}
\begin{tabular}{llll}\hline
Standard              &   DRAFT Percent   & Final \\ \hline\hline
Effective Figures             & 20\% &    & 10\% \\
Appropriate Trend Analysis    & 20\% &    & 10\% \\
Described Methods             & 20\% &    & 15\% \\
Peer-Reviewed Literature Effectively Discussed & 15\% & \%\\
Linkage to Climate Activists use and claims    & 50\% & \%\\
%Counter Arguments to Distorted Claims         & 20\% & \%\\
\hline
\end{tabular}
\end{table}



\subsection{Writing Reflection}

%\subsection{Peer Review Blogs}

\subsubsection{Rational}

Reflection is a tool to make something make something `visible' that is often invisible. As we develop our skills in science writing, reflection on the process of scientific writing will make it more likely that you successfully apply various skills and strategies to your future writing tasks. 

\subsubsection{Assignment}

Answer the following questions in less than 2 pages: 

\begin{itemize}
  \item How has the kind of feedback you give others about their writing or when talking science changed over during this project?

\item What was hardest aspect of the writing process for this project's products? What solutions (or ameliorations), if any, did you find for dealing with that? 

\item What advice would you give to an incoming EA30 student about how to write?

\item What aspect of the writing process that you learned or used in this course can you see taking with you to another philosophy class or to other places where you write and think? (the answer might be none, in which case, why do you think that is?)

\end{itemize}

\subsubsection{Submission Format and Naming Convention}

Write out your answers and save them using the following convention:

\begin{center}
WritingReflection\_XXXXX-Y.pdf
\end{center}

\noindent where XXXXX is one of your random numbers and Y is the project number. 

%\subsubsection{Reflection Grading}

%The peer review process will be graded using Table \ref{tab:blogpeerreviewgrading}. 

%\begin{table}[h]
%\caption{Blog Peer Review Grading Standards.}
%\label{tab:blogpeerreviewgrading}
%\begin{tabular}{llr}\hline
%Standard          &   Percent   & \\ \hline\hline
%Acknowledge specific blog  successes & 25\% \\
%Make concrete and detailed suggestions & 75\% \\
%\hline
%\end{tabular}
%\end{table}



\subsection{Op-Ed 2}

Coming soon! \smiley{}

%%\subsection{Op-Ed 2}

\subsubsection{Rational}

Successful editorials will include the following caracteristics: 1) a compelling and newsworthy opinion, 2) engagement with a controversy associated with Climate Change and 3) a mechanism to encourage the audience to read your blog. 

\subsubsection{Assignment}

Select a regional newspaper where you can submit your Op Ed. Learn the format and length allowed for a submission. 

Using the Op-Ed guidelines, write an Op-Ed to summarize 2-3 salient points from your Blog where you should:

\begin{itemize}
  \item Note how activists influence climate change narratives; 
  \item Describe regional climate changes and predictions that include ecological impacts; 
  \item Cite instances of how various scientific arguments have been distorted and politicized;
  \item Identify how conventional scientific standards have been compromised and how arguments that might be based on distortions can be countered.
\end{itemize}

Write and submit the editorial that highlights the key aspects of climate change that you discovered and link that to a newsworthy item. Be sure to cite and provide the URL to your blog. Submit the editorial and include the 'proof of receipt' with you submssion. 

\subsubsection{Submission Format and Naming Convention}

Uses the Op-Ed guidelines, submit a draft Op-Ed via \texttt{Sakai} and a separate document that describes the local or regional paper that this Op-Ed will be submitted to and several examples of Op-Eds that have discussed environmental issues in the paper.

Write an Op-Ed to propose what makes a good public product with respect to criticisms of climate science debates and criticisms. In other words, describe (2-3) ways that climate change skeptisism might misuse the data analysis and how one might prevent the misuse, be sure to cite your blog as an attepmpt to accomlish these goals. 

Submit Op-Ed to the appropriate regional or local paper and a copy as a PDF via \texttt{Sakai} with the following naming format: 

\begin{center}
LastName\_Region\_OpEd.pdf
\end{center}

\subsubsection{Grading of Published Op Ed 2}

The 2nd Op Ed will be grading using the criteria in Table \ref{tab:oped2grading}.

\begin{table}[h]
\centering
\caption{Published Op Ed Grading Criteria}
\label{tab:oped2grading}
\begin{tabular}{llp{3in}}\hline
Standard            & Percent \\ \hline\hline
Submission          & 10\% & Has a regional news source been identified? Has submission evidenced been provided?\\ 
Timeliness          & 20\% & Does the Op-Ed capture an event present in the news cycle? Does the event have a recognizable impact on readers? \\
Compelling          & 20\%  & Does the reader get drawn in with the first sentance?  Is there a pivot that maintains the readers interests? Does the reader get drawn into wanting to read the included blog link? \\
Themmatic     & 10\% & Does the Op-Ed address the semester's theme sufficiently?\\
Goal             & 20\% & Does the Op-Ed have a clear purpose? Is the goals obtained? \\
Succinct and Clear  & 20\% & Is the writing succinct, i.e. no sentences that detract from the goals and economical use of words.\\ \hline
\end{tabular}
\end{table}







\clearpage
\newpage
\subsection{DRAFT Blog -- Peer Evaluation}

\bigskip
Evaluator: \rule{7cm}{0.4pt}

\bigskip

\noindent Presenter: \rule{7cm}{0.4pt}

\begin{enumerate}
 \setlength\itemsep{4em}
  \item Describe two items you learned.
  \item Describe one concept or fact you would like to learn in more detail.
\end{enumerate}


\begin{table}[ht!]
\caption{Please circle the best response, where one is inadequate and five is outstanding---i.e. should be teaching the topic!}
\begin{tabular}{|p{4in}|ccccc|}\hline
How clear was the presentation?     & 1 & 2 & 3 & 4 & 5 \\ \hline
Suggestions: &&&&& \\ &&&&& \\ &&&&& \\
&&&&& \\ \hline
Did the analysis seem valid?        & 1 & 2 & 3 & 4 & 5 \\ \hline
Suggestions: &&&&& \\ &&&&& \\ &&&&& \\
&&&&& \\ \hline
Was information complete enough?            & 1 & 2 & 3 & 4 & 5 \\ \hline
Suggestions: &&&&& \\ &&&&& \\ &&&&& \\
&&&&& \\ \hline
To what extent could you use this example in climate discussions?            & 1 & 2 & 3 & 4 & 5 \\ \hline
Suggestions: &&&&& \\ &&&&& \\ &&&&& \\
&&&&& \\ \hline
\end{tabular}
\end{table}



\end{document}
