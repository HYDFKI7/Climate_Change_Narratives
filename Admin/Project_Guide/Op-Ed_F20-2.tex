%\subsection{Op-Ed 2}

\subsubsection{Rational}

Successful editorials will include the following caracteristics: 1) a compelling and newsworthy opinion, 2) engagement with a controversy associated with Climate Change and 3) a mechanism to encourage the audience to read your blog. 

\subsubsection{Assignment}

Select a regional newspaper where you can submit your Op Ed. Learn the format and length allowed for a submission. 

Using the Op-Ed guidelines, write an Op-Ed to summarize 2-3 salient points from your Blog where you should:

\begin{itemize}
  \item Note how activists influence climate change narratives; 
  \item Describe regional climate changes and predictions that include ecological impacts; 
  \item Cite instances of how various scientific arguments have been distorted and politicized;
  \item Identify how conventional scientific standards have been compromised and how arguments that might be based on distortions can be countered.
\end{itemize}

Write and submit the editorial that highlights the key aspects of climate change that you discovered and link that to a newsworthy item. Be sure to cite and provide the URL to your blog. Submit the editorial and include the 'proof of receipt' with you submssion. 

\subsubsection{Submission Format and Naming Convention}

Uses the Op-Ed guidelines, submit a draft Op-Ed via \texttt{Sakai} and a separate document that describes the local or regional paper that this Op-Ed will be submitted to and several examples of Op-Eds that have discussed environmental issues in the paper.

Write an Op-Ed to propose what makes a good public product with respect to criticisms of climate science debates and criticisms. In other words, describe (2-3) ways that climate change skeptisism might misuse the data analysis and how one might prevent the misuse, be sure to cite your blog as an attepmpt to accomlish these goals. 

Submit Op-Ed to the appropriate regional or local paper and a copy as a PDF via \texttt{Sakai} with the following naming format: 

\begin{center}
LastName\_Region\_OpEd.pdf
\end{center}

\subsubsection{Grading of Published Op Ed 2}

The 2nd Op Ed will be grading using the criteria in Table \ref{tab:oped2grading}.

\begin{table}[h]
\centering
\caption{Published Op Ed Grading Criteria}
\label{tab:oped2grading}
\begin{tabular}{llp{3in}}\hline
Standard            & Percent \\ \hline\hline
Submission          & 10\% & Has a regional news source been identified? Has submission evidenced been provided?\\ 
Timeliness          & 20\% & Does the Op-Ed capture an event present in the news cycle? Does the event have a recognizable impact on readers? \\
Compelling          & 20\%  & Does the reader get drawn in with the first sentance?  Is there a pivot that maintains the readers interests? Does the reader get drawn into wanting to read the included blog link? \\
Themmatic     & 10\% & Does the Op-Ed address the semester's theme sufficiently?\\
Goal             & 20\% & Does the Op-Ed have a clear purpose? Is the goals obtained? \\
Succinct and Clear  & 20\% & Is the writing succinct, i.e. no sentences that detract from the goals and economical use of words.\\ \hline
\end{tabular}
\end{table}




