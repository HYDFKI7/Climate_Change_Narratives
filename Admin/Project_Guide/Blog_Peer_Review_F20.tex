%\subsection{Peer Review Blogs}

\subsubsection{Rational}

Reviewing a public product is a priviledge. And for the `reviewed' it's a gift. Thus, for each, the reviewer and reviewed, the value for the greater good is indisputable. 

As you review your collegeues work, try to keep in mind that you are promoting a better outcome and better science. In addition, pay attention to thinks that might have escaped your own process and that you find yourself saying, ``wow, that's a cool approach!''  Perhaps, you might adapt some of the things you read into your own writing!

\subsubsection{Assignment}

To assess the Blogs, each student will review three blogs and submit a evaluation form for each one.

\subsubsection{Submission Format and Naming Convention}

Fill out the forms and save them using the following convention:

\begin{center}
BlogAuthorSurname\_XXXXX.pdf
\end{center}

\noindent after submitting the review to \texttt{Sakai}, I will make each available to the Blog authors to help each us revise and improve our blogs. 

\subsubsection{Blog Peer Review Grading}

The peer review process will be graded using Table \ref{tab:blogpeerreviewgrading}. 

\begin{table}[h]
\caption{Blog Peer Review Grading Standards.}
\label{tab:blogpeerreviewgrading}
\begin{tabular}{llr}\hline
Standard          &   Percent   & \\ \hline\hline
Acknowledge specific blog  successes & 25\% \\
Make concrete and detailed suggestions & 75\% \\
\hline
\end{tabular}
\end{table}

