%\subsection{Peer Review Blogs}

\subsubsection{Rational}

Although writing seen as an individual process -- in reality it's a very social activity, at minumum we don't just write for ourselves. Moreover, as a public product, we want our blogs to be understood and appreciated by a wide audience. 

Reviewing a public product is a priviledge. And for the `reviewed' it's a gift. Thus, for each, the reviewer and reviewed, the value for the greater good is indisputable. 

As you review your collegeues work, try to keep in mind that you are promoting a better outcome and better science. In addition, pay attention to thinks that might have escaped your own process and that you find yourself saying, ``wow, that's a cool approach!''  Perhaps, you might adapt some of the things you read into your own writing!

\subsubsection{Assignment}

To assess the Blogs, each student will review three blogs and submit a evaluation form for each one.

As you review the blog, determine the target audience and evaluat the blog based on your perception of this target. The author may not have the same target in mind, but identifying the divergence might be very helpful. 

Evaluate what if the questions was clearly identified, determine if the data were analyzed clearly (hypothesis, statistical tests), and decide if the conclusions were clear. In addition, make suggestions about how each blog could be improved.

Finally, as we can all appreciate, we are often our own worst readers when it comes to finding issues with grammar and style -- typos often hard to detect and grammar issues may be invisible because we might ``correct'' the text in our brain. Thus, as you read your colleagues blogs and try to identify these issues in the text. Usually, we can do this using Microsoft Word using ``Track Changes''. However, with differential access to Microsoft products, we will have to be creative. If you do not have access to Microsoft, we will work with you to come up with ways to give detailed comments without using this software.

The form will be available on Sakai

\subsubsection{Submission Format and Naming Convention}

Fill out the forms and save them using the following convention:

\begin{center}
BlogAuthorSurname\_XXXXX.pdf
\end{center}

\noindent after submitting the review to \texttt{Sakai}, I will make each available to the Blog authors to help each us revise and improve our blogs. 

\subsubsection{Blog Peer Review Grading}

The peer review process will be graded using Table \ref{tab:blogpeerreviewgrading}. 

\begin{table}[h]
\caption{Blog Peer Review Grading Standards.}
\label{tab:blogpeerreviewgrading}
\begin{tabular}{llr}\hline
Standard          &   Percent   & \\ \hline\hline
Acknowledge specific blog  successes & 25\% \\
Make concrete and detailed suggestions & 75\% \\
\hline
\end{tabular}
\end{table}

