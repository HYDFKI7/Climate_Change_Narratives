%\subsection{Regional Climate Impacts -- Literature Review}

\subsubsection{Rationale}

By using peer reviewed literature, we can assess our regional analysis to determine if the are trends that have been predicted that align or possibly contradict our analysis. 

This assignment is designed to help you put create a blog and have compelling {\color{red}and scientifically rooted information}. 

\subsubsection{Assignment}

Review regionally relative results and conclusions from peer reviewed climate science. %See this document as a resource.


Evaluate peer-reviewed articles to determine potential ecological, economic, and sociological implications of climate patterns. We suggest that you find 3-5 peer review literature articles, but you might fin{\color{red}d} useful government or NGO reports can also be added. {\color{red}In general, peer reviewed literature is considered the best source of scientific information.} 

Summarize these papers into a stand-alone paper. 

\subsubsection{Submission Format and Naming Convention}

The paper should be double-space, 12 point font, and less than 5 pages (excluding citations). As a pdf, the paper should be submitted via Sakai with the following naming convention:

\begin{center}
RegionalImpacts\_F20\_XXXXX.pdf
\end{center}

\medskip \noindent where the XXXXX refer to one set of the assigned random numbers. 

\subsubsection{Grading of the Regional Impacts Summary}

The regional impacts review will be grading using the criteria in Table \ref{tab:regionalimpactsgrading}.

\begin{table}[h]
\caption{Summary of Data Analysis grading standards.}
\label{tab:regionalimpactsgrading}
\begin{tabular}{llc}\hline
Criteria            &   Standard    & Percent \\ \hline\hline
Sources     & Compelling, e.g. Over 5 peer reviewed papers; correctly cited & 15\% \\
Ecological  & Knowns and unknowns             & 20\% \\
Economic    & costs and benefits              & 20\% \\
Social      & e.g. Social Justice             & 20\% \\
Communication    & Accurate, e.g. rejected null   & 25\% \\

\hline
\end{tabular}
\end{table}