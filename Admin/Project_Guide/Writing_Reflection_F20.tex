%\subsection{Peer Review Blogs}

\subsubsection{Rational}

Reflection is a tool to make something make something `visible' that is often invisible. As we develop our skills in science writing, reflection on the process of scientific writing will make it more likely that you successfully apply various skills and strategies to your future writing tasks. 

\subsubsection{Assignment}

Answer the following questions in less than 2 pages: 

\begin{itemize}
  \item How has the kind of feedback you give others about their writing or when talking science changed over during this project?

\item What was hardest aspect of the writing process for this project's products? What solutions (or ameliorations), if any, did you find for dealing with that? 

\item What advice would you give to an incoming EA30 student about how to write?

\item What aspect of the writing process that you learned or used in this course can you see taking with you to another philosophy class or to other places where you write and think? (the answer might be none, in which case, why do you think that is?)

\end{itemize}

\subsubsection{Submission Format and Naming Convention}

Write out your answers and save them using the following convention:

\begin{center}
WritingReflection\_XXXXX-Y.pdf
\end{center}

\noindent where XXXXX is one of your random numbers and Y is the project number. 

%\subsubsection{Reflection Grading}

%The peer review process will be graded using Table \ref{tab:blogpeerreviewgrading}. 

%\begin{table}[h]
%\caption{Blog Peer Review Grading Standards.}
%\label{tab:blogpeerreviewgrading}
%\begin{tabular}{llr}\hline
%Standard          &   Percent   & \\ \hline\hline
%Acknowledge specific blog  successes & 25\% \\
%Make concrete and detailed suggestions & 75\% \\
%\hline
%\end{tabular}
%\end{table}

