%\subsection{Writing a Scientific Blog}

\subsubsection{Rational}

{\color{red}Writing our blogs give us the opportunity to evaluate the climate trends in a specific region -- importantly, we are writing the blogs to explain historic and predicted climate change impacts to a specific audience (of your choosing). Thus, we will have to 1) understand climate science enough to explain it, and 2) translate the impacts into something that your readers will appreciate. Thus, as in many environmental issues, learning the scientific issues and figuring out how to communicate these requires patience and practice.


\subsubsection{Developing a Narrative}

Linking your ``Regional Analysis'' and ``Regional Climate Science Literature Review'', compose a blog with the following characteristics: 

\begin{itemize}
  \item focused narrative the engages an audience that is invested in the region analyzed; 
  \item figures that demonstrate climate trends using R and report (and explain) key statistics; and  
  \item cite peer review literature as additional sources of evidence.
\end{itemize}

For the 2020 blog, we want to link these blogs to ``climate activist''. We are using this term as losely as we can. Activitist might include politicians, business leaders, NGO spokepersons, and, of course, activitists at-large. I wonder if we might scour social media as a possible source as well, but as the project develops, we will see what avenues are fruitful. These are your projects, so I suspect I will be surprised by directions you decide to take. 

\subsubsection{Additional Suggestions}

Write blog to effectively and clearly describe results. The blog shall be publish-ready and include the following: 

\begin{itemize*}
  %\item Appropriate and thoughtful statistical analysis;
  \item Describe the economic, cultural, and physical geography of the region (2-3 sentences);
  \item Describe typical climate patterns (1-2 sentances);
  \item Describe where the data were obtained and summarize how the data were processed and analzyed;
  \item Time series plots of temperture data using R (1-3 graphs, with several setences describing the results) as needed by the narrative;
  \item Evaluation of data to determine if trends exists;
  \item Compare results to model predictions/forecasts and possible ecological and economic implications to the region; 
    \item description of what the data tells about about the region, 
  \item a few sentances describing how data can be interpretted; pitfalls of unintentional and intentional misinterpretations; and 
  \item narrative that describes the climate and climate change implications for a community that you care about.
  %\item Describe how the data should be presented, e.g. how the data should be interpreted, and how to avoid misinterpretations that are present in the popular culture.
\end{itemize*}

\subsubsection{Draft and Revision Process}

Science writing (all writing) is a social process. In the case of writing about climate change, we rely on a pretty technical language, which means translating this language for others is not trivial. 

To help in the process, we will turn in a draft blog that will be evaluated by your peers and faculty with the aim to help you focus your thesis, ensure you have described the methods and results effectively, include the evidenced need, and finally in a manner that a broad audience can appreciate. 

\subsubsection{Submission Format and Naming Convention}

To facilite the publishing the blogs, please save the Rmd and html using the following conventions: 

\begin{center}
Surname.Rmd
\end{center}

\noindent Knit the file using the option to create a word document. Upload this document into \texttt{Sakai} on the due date. 

\subsubsection{Scientific Blog Grading}

The Blogs will be grading using Table \ref{tab:bloggrading}. NOTE: We will focus on the mechanics of the blog for the first draft to ensure we have a good foundation on the data and some leaway as we refine our narratives. 

}
