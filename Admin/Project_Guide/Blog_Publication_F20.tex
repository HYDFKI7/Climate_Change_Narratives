%\subsection{Publishing Revised Blog}

\subsubsection{Rational}

Our capacity to publish our blogs demonstrates that our projects have value beyond our classroom. In addition, these provide a litmus test for our work -- how will the public or specific stakeholders respond to our efforts. Will they see this a valueable, value-added, or problematic?  Although we might not get immediate feedback, the process to publish our blogs gives an opportunity that would be missing if we only wrote papers for the instructor!

\subsubsection{Assignment}

Capitalizing on the ``regional data analysis'' and ``regional climate science literture review'', revise your blog to address peer review and faculty comments.

To facilate access to blogs and encourage readers, we will create short `hooks'. As a teaser, the hook is like a `elevator pitch', summarizing salient points that links the index page to your own blog. The hook should only be one (perhaps two) sentence(s). 

%\begin{itemize*}
%  \item Effectively display climate patterns from NOAA repositories, with at least 6 decades of data. Be sure all graphics are appropriate labeled and have captions that the reader can use to intrepret the data;
%  \item Analyze the data using a linear model using R (i.e. lm);
%  \item Describe the methods used to obtain and analyze the data; and
%  \item Evaluate peer review literature to determine potential regional impacts from climate change -- be sure to include ecological and economic impacts; 
%  \item Cite instances of how various scientific arguments have been distorted and politicized;
%  \item Identify how conventional scientific standards have been compromised and how arguments that might be based on distortions can be countered.
%\end{itemize*}

%If it helps, read the Project\_Report.pdf on the Project Site for some helpful hints.

\subsubsection{Submission Format and Naming Convention}

The Blog will be published online (via \url{Github.com}), using the following naming convention: 

\begin{center}
Surname.Rmd and Surname.html
\end{center}

\subsubsection{Published Blog Grading}

The Blogs will be grading using Table \ref{tab:bloggrading}. 

\begin{table}[h]
\caption{Climate Science Blog Grading criteria and percentages.}
\label{tab:bloggrading}
\begin{tabular}{lrr}\hline
Standard              &   DRAFT   & Final \\ \hline\hline
Regional issues introduced     & 20\%    & 10\% \\
Described methods             & 20\%    & 5\% \\
Effective figures             & 20\%    & 10\% \\
Appropriate trend analysis    & 20\%    & 10\% \\
Peer-Reviewed literature effectively discussed & 10\% & 30\%\\
Linkage to climate activists use and claims    & 10\% & 30\%\\
%Counter Arguments to Distorted Claims         & 20\% & \%\\
Blog Hook                                      & 0\%  & 5\%\\
\hline
\end{tabular}
\end{table}

%Here's a checklist that might be useful to check based on observations we have made for the drafts:


