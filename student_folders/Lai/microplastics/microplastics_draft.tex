\documentclass[]{article}
\usepackage{lmodern}
\usepackage{amssymb,amsmath}
\usepackage{ifxetex,ifluatex}
\usepackage{fixltx2e} % provides \textsubscript
\ifnum 0\ifxetex 1\fi\ifluatex 1\fi=0 % if pdftex
  \usepackage[T1]{fontenc}
  \usepackage[utf8]{inputenc}
\else % if luatex or xelatex
  \ifxetex
    \usepackage{mathspec}
  \else
    \usepackage{fontspec}
  \fi
  \defaultfontfeatures{Ligatures=TeX,Scale=MatchLowercase}
\fi
% use upquote if available, for straight quotes in verbatim environments
\IfFileExists{upquote.sty}{\usepackage{upquote}}{}
% use microtype if available
\IfFileExists{microtype.sty}{%
\usepackage{microtype}
\UseMicrotypeSet[protrusion]{basicmath} % disable protrusion for tt fonts
}{}
\usepackage[margin=1in]{geometry}
\usepackage{hyperref}
\hypersetup{unicode=true,
            pdftitle={Comparing Microplastic Pollution in Romaine Lettuce Sold With or Without Plastic Packaging},
            pdfauthor={bljj2015},
            pdfborder={0 0 0},
            breaklinks=true}
\urlstyle{same}  % don't use monospace font for urls
\usepackage{longtable,booktabs}
\usepackage{graphicx,grffile}
\makeatletter
\def\maxwidth{\ifdim\Gin@nat@width>\linewidth\linewidth\else\Gin@nat@width\fi}
\def\maxheight{\ifdim\Gin@nat@height>\textheight\textheight\else\Gin@nat@height\fi}
\makeatother
% Scale images if necessary, so that they will not overflow the page
% margins by default, and it is still possible to overwrite the defaults
% using explicit options in \includegraphics[width, height, ...]{}
\setkeys{Gin}{width=\maxwidth,height=\maxheight,keepaspectratio}
\IfFileExists{parskip.sty}{%
\usepackage{parskip}
}{% else
\setlength{\parindent}{0pt}
\setlength{\parskip}{6pt plus 2pt minus 1pt}
}
\setlength{\emergencystretch}{3em}  % prevent overfull lines
\providecommand{\tightlist}{%
  \setlength{\itemsep}{0pt}\setlength{\parskip}{0pt}}
\setcounter{secnumdepth}{0}
% Redefines (sub)paragraphs to behave more like sections
\ifx\paragraph\undefined\else
\let\oldparagraph\paragraph
\renewcommand{\paragraph}[1]{\oldparagraph{#1}\mbox{}}
\fi
\ifx\subparagraph\undefined\else
\let\oldsubparagraph\subparagraph
\renewcommand{\subparagraph}[1]{\oldsubparagraph{#1}\mbox{}}
\fi

%%% Use protect on footnotes to avoid problems with footnotes in titles
\let\rmarkdownfootnote\footnote%
\def\footnote{\protect\rmarkdownfootnote}

%%% Change title format to be more compact
\usepackage{titling}

% Create subtitle command for use in maketitle
\providecommand{\subtitle}[1]{
  \posttitle{
    \begin{center}\large#1\end{center}
    }
}

\setlength{\droptitle}{-2em}

  \title{Comparing Microplastic Pollution in Romaine Lettuce Sold With or Without
Plastic Packaging}
    \pretitle{\vspace{\droptitle}\centering\huge}
  \posttitle{\par}
    \author{bljj2015}
    \preauthor{\centering\large\emph}
  \postauthor{\par}
      \predate{\centering\large\emph}
  \postdate{\par}
    \date{4/22/2019}


\begin{document}
\maketitle

\hypertarget{methods}{%
\subsubsection{Methods}\label{methods}}

The following methods are based off of prior established methods for
obtaining microplastic counts from food products (Löder et al.~2017,
Maes et al.~2017, Wang and Wang 2018, Karlsson et al.~2017, and Quinn et
al.~2018).

We purchased two unpackaged heads of romaine lettuce and two bags of
romaine lettuce in plastic packaging from each of the chosen five stores
in the cities of Claremont and Pomona. We chose only stores that sell
both unpackaged heads AND packaged lettuce (lettuce packaged as chopped
salad is acceptable). This led us to select the following supermarkets:
Sprouts on Foothill Blvd, State Bros.~on N Garey Ave, Cardenas on E Holt
Ave, El Super on E Holt Ave, and Super King on Auto Center Dr.~We put
the unpackaged romaine lettuce into paper bags to minimize contamination
during transit.

In the lab, we used a glass blender to homogenize the lettuce with
water. All equipment in contact with lettuce were first thoroughly
washed with Milli-Q deionized water for each successive sample. For each
sample of unbagged lettuce, we washed the head in Milli-Q deionized
water, then tore out leaves and inserted them into the blender until it
reached the top. Bagged lettuce was not washed, since they are labeled
``pre-washed'' and we wanted to test the difference. Bagged lettuce was
poured straight from the bag into the blender until it reached the top.
Afterwards, we poured in 200 mL of Milli-Q deionized water into the
blender. We blended the mix of lettuce and water for 30 seconds under
the slower ``aaaa'' setting, then 60 seconds under the faster ``bbbb''
setting. We strained the homogenized lettuce and water through a 5 mm
stainless steel sieve into a glass beaker to obtain 100 mL of solution.
This process was repeated for each sample, for a total of 20
experimental samples (two unbagged, two bagged from each of the five
stores).

To break down the cellulose in the lettuce solution, we added 5 mL
cellulase from \emph{Aspergillus niger} to separate particles in the
solution by density, we added 10 mL phosphate-buffered saline (PBS) to
each beaker. 1 L of phosphate-buffered saline solution was prepared
using 8 g sodium chloride (NaCl), 200 mg potassium chloride (KCl), 1.44
g disodium phosphate (Na\textsubscript{2}HPO\textsubscript{4}), and 240
mg monopotassium phosphate (KH\textsubscript{2}PO\textsubscript{4})
prepared in deionized water, then set to pH 5 using hydrochloric acid
(HCl). Three blank control samples were prepared using 100 mL Milli-Q
deionized water in a glass beaker. The beakers were then covered with
aluminium foil wrap and incubated at 50°C for four days.

After incubation, we added 50 mL NaCl solution to each beaker (density =
1.2 g NaCl / 1 mL water, 1440 mL solution stirred in 2 L volumetric
flask for 10 minutes). For each beaker, 5 mL of 0.08 g/mL Red Nile dye
solution was added and allowed to stain the sample solution for 30
minutes. For each of the 23 total samples, 40 mL of solution was
extracted from the top of the beaker using a glass vacuum and glass
pipette through a sheet of filter paper. The filter paper was placed in
glass containers and covered with aluminium foil to let dry overnight.
We then used a digital microscope with fluorescent light (red
fluorescent 100\% brightness, 110 ms capture) to count stained
microplastic particles next to four randomly generated points at XXX
zoom for each sample of filter paper. Four microplastic counts were
obtained for each of the 20 experimental samples and 3 blank samples.

Statistical analysis was conducted using the program R (CRAN 2019). In
the analysis, we paired the average of the two unbagged counts with the
average of the two bagged counts for each of the five stores to test if
there was a statistically significant difference between the two groups.
The parametric t-test (for normal distribution) and nonparametric
Wilcoxon signed-rank test (for non-normal distribution) were used to
obtain the p-value.

\hypertarget{results}{%
\subsubsection{Results}\label{results}}

\includegraphics{microplastics_draft_files/figure-latex/graphs for all lettuce data-1.pdf}

\emph{\textbf{Figure 1.} Boxplot of Microplastic Counts}

\begin{longtable}[]{@{}lll@{}}
\toprule
Test & p-value & Statistical significance?\tabularnewline
\midrule
\endhead
1. paired t-test & 0.38 & no\tabularnewline
2. Wilcoxon signed-rank test & 0.58 & no\tabularnewline
\bottomrule
\end{longtable}

\emph{\textbf{Table 1.} Test of differences between bagged and unbagged
microplastic counts paired with each store}

We did both the parametric and non-parametric paired tests because there
was one outlier in each of the unbagged and bagged averages that caused
the distribution to be non-normal (Figure 1). We do not reject the null
hypothesis based on the lack of statistical significance in the paired
t-test and Wilcoxon signed-rank test (Table 1). This indicates there is
no significant difference in microplastic count between unbagged heads
of romaine lettuce and romaine lettuce bagged in plastic packaging.

\hypertarget{discussion-in-progress}{%
\subsubsection{Discussion (in progress)}\label{discussion-in-progress}}

issues: * inevitable airborne contamination in transit, in lab might
have affected results * hard to identify microplastics from lettuce
fibres even after cellulose removed using cellulase


\end{document}
